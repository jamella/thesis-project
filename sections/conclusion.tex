Due to the limitations of \ac{IoT} devices, achieving secure communications is not an easy task. In order to allow the deployment of battery powered nodes, their communication model must be very efficient and consume the minimum amount of power required for operation. To achieve those requirements we started by analysing the existing protocols across the OSI layers, trying to find the best suited solutions for this type of environments. After a thorough comparison we achieved a working stack of protocols but soon discovered possible breaches and attacks, specially on the network layer. Those attacks were further investigated and catalogued, providing an overview of the attacks spectrum. Given the common principle on the majority of the attacks, the introduction of rogue nodes to the network, we presented some possible solutions based on secure bootstrapping, the secure authentication of new nodes when joining a network.
Once the energy efficient stack, possible attacks and mitigation strategies were defined, we proposed our solution based on a Smart Campus scenario. This solution is focused on providing the joining devices all the secure credentials required for a secure bootstrapping before the deploy on the field, so that when they start the operation phase no additional credentials need to be fetched, implying that no additional energy is spent on configuration.
Always maintaining a power-aware perspective, the system will be evaluated by measuring its energy consumption with different configurations. These range from no security where messages are sent in plain text and no node authentication is performed, to full security, where the node is authenticated at network and application layers and messages are sent cyphered. This charting allows future users of the system to decide the type of resources they need to allocate in order to achieve a desired level of security for their application. As future work, currently out of the scope of this project, memory access protection should be addressed in order to the prevent the stealing of secure credentials from deployed devices.