\paragraph{}The \ac{IoT} can be seen as web of interconnected devices that go from everyday wearable objects into fully deployed sensor networks.Despite the huge variety and characteristics of these devices, one thing that they all have in common in the constrained nature they're built upon. In order to enable the massive deploy to be expected in the near future \footnote{http://blogs.wsj.com/cio/2015/06/02/internet-of-things-market-to-reach-1-7-trillion-by-2020-idc/} \ac{IoT} devices must be accessible and affordable, capable of operating under lossy wireless networks while being battery powered. This poses a challenge to current \ac{WWW} protocols since the assumptions regarding the devices capabilities and objectives do not hold true. To allow the \ac{IoT} vision to come forward, several new protocols have been developed across the OSI layers, each addressing and tackling the challenges involved in trying to keep the quality and assurances of stronger, more expensive protocols, on constrained systems. This document will address this issue from a power-aware perspective, meaning the battery consumption will be of major importance.
