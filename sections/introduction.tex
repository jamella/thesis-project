\paragraph{}The Internet of Things \ac{IoT} can be seen as a web of interconnected devices that go from everyday wearable objects into fully deployed sensor networks. Despite the huge variety and characteristics of these devices, one thing that they all have in common is the constrained nature that they are built upon. In order to enable the massive deployment to be expected in the near future,\footnote{http://blogs.wsj.com/cio/2015/06/02/internet-of-things-market-to-reach-1-7-trillion-by-2020-idc/} \ac{IoT} devices must be accessible and affordable, capable of operating under lossy wireless networks while being battery powered. This poses a challenge to current Internet protocols since the assumptions regarding the devices' capabilities and objectives do not hold true.\\ To allow the \ac{IoT} vision to come forward, several new protocols have been developed across the OSI layers, each addressing and tackling the challenges involved in trying to keep the quality and assurances of stronger, more expensive protocols, on constrained systems.
Additionally, security is also a very important due to the fact that the interconnection of the devices around us can provide information about our choices and whereabouts, therefore reducing our privacy.
This document will address these issues from a power-aware perspective, meaning that the battery consumption will be of major importance.

\subsection{Main Goals}

\paragraph{}
Given the constraints and limitations of \ac{IoT} devices described in the previous section and the recent protocols, the first objective of this work is to identify a working stack of protocols that takes into account the \ac{IoT} constraints and focuses on allowing a power-aware communication model.

\paragraph{}
After the analysis of the existing solutions, a baseline of power consumption will be established. Then, the focus will move towards adding authentication, confidentiality and integrity to the transmitted information by securing the channel.
Once both the application level protocols and proper security solutions are defined, experiments will be performed so that the added power consumption cost of adding the security layers can be measured, profiled and documented therefore enabling the finding of the best parameters for a desired level of security.

\paragraph{}
The work will then proceed towards finding effective counter-measures against a specific group of attacks that targets the \ac{IoT} devices by intensifying the use of its resources, therefore draining the available power and placing the device offline. The ultimate goal is to propose an energy-efficient administration system that can provide the tools to resist those attacks and assure a proper working network.

\subsection{Document Roadmap}

In this document we start by analysing the state-of-the-art in Section \ref{sec:related_work}. This includes the selection of the most adequate protocol stack for our necessities in Section \ref{sec:protocol_analysis}, an overview of the existing attacks and mitigation strategies in Section \ref{sec:attack_analysis} and a summary of the existing solutions regarding secure insertion of new nodes in an existing network in Section \ref{sec:secure_bootstrapping}. All this knowledge will be integrated into our proposed solution defined in Section \ref{sec:proposed_solution}. Section \ref{sec:work_evaluation} defines how our work will be tested and evaluated so that a power-aware perspective can be achieved. Section \ref{sec:work_planning} states how the development of our solution will unwind over the next months and finally, Section \ref{sec:conclusion} presents the conclusion of this document.