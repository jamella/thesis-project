Following the power-aware perspective of this work, our solution will evaluate the power consumption of the system in several scenarios with different network configurations:

\begin{itemize}
	\item No Security: The system does not provide any type of security credentials. All messages are exchanged in plain text and no node authentication is performed.
	\item Shared Key: The system provides to new nodes a shared group key that enables them to join a secure instance of the network layer protocol RPL therefore assuring node authentication at the network layer.
	\item Asymmetric Cryptography : The system provides to new nodes an asymmetric key pair and the client observer public key. This enables the new nodes to join a secure instance of the application layer protocol \ac{CoAP} therefore assuring node authentication at the application layer. Moreover, this enables the \ac{DTLS} handshake to be performed using raw public keys assuring message confidentiality and integrity.
	\item Full Security Credentials: The system provides to new nodes both the shared group key, the asymmetric key pair and the client observer public key. 
\end{itemize}

After the data is collected, it will be analysed and charted so that the added power consumption of inserting security measures can be traced to the increasing power consumption. This will allow a network administrator to consider the type of device and powering mechanisms to deploy based on the security level he desires for a given application.